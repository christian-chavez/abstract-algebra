\def\thetitle{Homework 2}
\pagebreak

\begin{center}
    \includegraphics[height=0.075\textheight]{images/LogoYachay.pdf} 
    \hspace{0.1\linewidth}
    \includegraphics[height=0.075\textheight]{images/LogoECMC.pdf}
\end{center}

% \phantomsection\pdfbookmark{\currfilebase}{\currfilebase}

\begin{center}
    {\LARGE
    Abstract Algebra 2024--I\\
    % Teaching Assistant
    \vspace{0.25cm}
    \textbf{\thetitle{}}}

    % \emph{I hear, I forget;    I see, I remember;     I do, I understand.}
    
    Pablo Rosero \& Christian Chávez
    
    % \today

    September 11, 2023
\end{center}

% \setcounter{problem}{0}


\begin{questions}
\question Determine which of the following binary operations are associative.
\begin{enumerate}[label=(\alph*)]
    \item the operation \(\star\) on \(\Z\) defined by \(a\star b  = a-b\)
    \item the operation \(\star\) on \(\R\) defined by \(a\star b  = a+b+ab\)
    \item the operation \(\star\) on \(\Q\) defined by \(a\star b  = \dfrac{a+b}{5}\)
    \item the operation \(\star\) on \(\Z\times\Z\) defined by \((a , b) \star (c,d)  = (ad+bc, bd)\)
    \item the operation \(\star\) on \(\Q\backslash\left\{ 0 \right\}\) defined by \(a\star b = \dfrac{a}{b} \)
\end{enumerate}

\question 
Prove that addition of residue classes in \(\Z / n \Z\) is associative. (Assume it is well defined.)


\question
Determine which of the following sets are groups under addition:
\begin{enumerate}[label=(\alph*)]
    \item the set of rational numbers (including \(0=0 / 1\)) in lowest terms whose denominators are odd
    \item the set of rational numbers (including \(0=0 / 1\)) in lowest terms whose denominators are even
    \item the set of rational numbers of absolute value \(<1\)
    \item the set of rational numbers of absolute value \(\geq 1\) together with 0
    \item the set of rational numbers with denominators equal to 1 or 2
    \item the set of rational numbers with denominators equal to 1, 2 or 3 
\end{enumerate}



\question
Let \(G=\left\{z \in \mathbb{C} \mid z^n=1 \text{ for some }n \in \mathbb{Z}^{+}\right\}\).
\begin{enumerate}[label=(\alph*)]
    \item Prove that \(G\) is a group under multiplication (called the group of \textit{roots of unity} in \(\mathbb{C}\)).
    \item Prove that \(G\) is not a group under addition.
\end{enumerate}


\question
Let \(G=\{a+b \sqrt{2} \in \mathbb{R} \mid a, b \in \mathbb{Q}\}\).
\begin{enumerate}[label=(\alph*)]
    \item Prove that \(G\) is a group under addition.
    \item Prove that the nonzero elements of \(G\) are a group under multiplication. (``Rationalize the denominators'' to find multiplicative inverses.)
\end{enumerate}


\question
Find the orders of each element of the additive group \(\mathbb{Z} / 12 \mathbb{Z}\).

\question
Find the orders of the following elements of the multiplicative group \((\mathbb{Z} / 12 \mathbb{Z})^\times\):
\[\overline{1}, \overline{-1}, \overline{5}, \overline{7}, \overline{-7}, \overline{13}.\]

\question
Find the orders of the following elements of the additive group \(\mathbb{Z} / 36 \mathbb{Z}\):
\[ \overline{1}, \overline{2}, \overline{6}, \overline{9}, \overline{10}, \overline{12}, \overline{-1},\overline{-10},\overline{-18}.\]



\question
Let \(x\) be an element of \(G\). Prove that \(x^2=1\) if and only if \(|x|\) is either 1 or 2.

\question
Let \(x\) be an element of \(G\). Prove that if \(|x|=n\) for some positive integer \(n\) then \(x^{-1}=x^{n-1}\).

\question
Let \(x\) and \(y\) be elements of \(G\). Prove that \(x y=y x\) if and only if \(y^{-1} x y=x\) if and only if \(x^{-1} y^{-1} x y=1\).


\question
Let \(x \in G\) and let \(a, b \in \mathbb{Z}^{+}\).
\begin{enumerate}[label=(\alph*)]
    \item Prove that \(x^{a+b}=x^a x^b \) and \(\left(x^a\right)^b=x^{a b}\).
    \item Prove that \(\left(x^a\right)^{-1}=x^{-a}\).
    \item Establish part (a) for arbitrary integers \(a\) and \(b\) (positive, negative or zero).
\end{enumerate}

\question
For \(x\) an element in \(G\) show that \(x\) and \(x^{-1}\) have the same order.


\question 
If \(x\) and \(g\) are elements of the group \(G\), prove that \(|x|=\left|g^{-1} x g\right|\). Deduce that \(|a b|=|b a|\) for all \(a, b \in G\).

\question
 Prove that if \(x^2=1\) for all \(x \in G\), then \(G\) is abelian.

\question
Assume \({H}\) is a nonempty subset of \((G, \star)\) which is closed under the binary operation on \(G\) and is closed under inverses, i.e., for all \(h\) and \(k\) elements of \(H\) it holds    \(hk,h^{-1} \in H\). Prove that \(H\) is a group under the operation \(\star\) restricted to \(H\) (such a subset \(H\) is called a subgroup of \(G\) ).

\question
Prove that if \(x\) is an element of the group \(G\) then \(\left\{x^n \mid n \in \mathbb{Z}\right\}\) is a subgroup (cf. the preceding exercise) of \(G\) (called the cyclic subgroup of \(G\) generated by \(x\)).

\question
Compute the order of each of the elements in  (a) \(D_6\), (b) \(D_8\), and (c) \(D_{10}\).



\question
Let \(\sigma\) be the permutation
\[
1 \mapsto 3 \quad 2 \mapsto 4 \quad 3 \mapsto 5 \quad 4 \mapsto 2 \quad 5 \mapsto 1
\]
and let \(\tau\) be the permutation
\[
1 \mapsto 5 \quad 2 \mapsto 3 \quad 3 \mapsto 2 \quad 4 \mapsto 4 \quad 5 \mapsto 1 .
\]

Find the cycle decompositions of each of the following permutations: \(\sigma, \tau, \sigma^2, \sigma \tau, \tau \sigma\), and \(\tau^2 \sigma\).


% Quedo en la página 4 del PDF




\question







\question









\question







\question








\question







\question











\question










\question










\question



















































\end{questions}